\cvsection{Research\\Experience}{

    \cvitem{LSST Dark Energy Science Collaboration (DESC)}{2019-present}{
        Leading the high-redshift cosmology analysis using Lyman-break Galaxies (LBGs), including measurements of the UV Luminosity Function, clustering, and cross-correlations with CMB lensing.
        Also developing the photometric redshift pipeline for DESC cosmology.
    }
    
    \cvitem{The Vera C. Rubin Observatory}{2021-present}{
        Developing and commissioning the active optics system, including leading development of wavefront estimation algorithms, using analytic, forward modeling, and deep learning methods.
        Member of the galaxy photometry and photometric redshift (photo-z) commissioning teams and the observing support team.
    }
    
    \cvitem{Duke University Neutrino and Cosmology Group}{2016-2019}{
        Simulated core-collapse supernova neutrino bursts.
        Quantified sensitivity and developed Bayesian analysis methods for the Helium and Lead Observatory (HALO) neutrino detector.
    }

    \cvitem{Karlsruhe Institute of Technology}{2018}{
        Studied muon content of cosmic rays detected with the IceTop Array and developed deep learning methods for data analysis (advised by Andreas Haungs).
    }

}