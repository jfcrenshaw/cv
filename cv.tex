%%%%%%%%%%%%%%%%%%%%%%%%%%%%%%%%%%%%%%%%%%%%%%%%%%%%%%%%%%%%%%%%%%%%%%%%%%%%%%%%
% Medium Length Graduate Curriculum Vitae
% LaTeX Template
% Version 1.2 (3/28/15)
%
% This template has been downloaded from:
% http://www.LaTeXTemplates.com
%
% Original author:
% Rensselaer Polytechnic Institute 
% (http://www.rpi.edu/dept/arc/training/latex/resumes/)
%
% Modified by:
% Daniel L Marks <xleafr@gmail.com> 3/28/2015
%
% Important note:
% This template requires the res.cls file to be in the same directory as the
% .tex file. The res.cls file provides the resume style used for structuring the
% document.
%
%%%%%%%%%%%%%%%%%%%%%%%%%%%%%%%%%%%%%%%%%%%%%%%%%%%%%%%%%%%%%%%%%%%%%%%%%%%%%%%%

%-------------------------------------------------------------------------------
%	PACKAGES AND OTHER DOCUMENT CONFIGURATIONS
%-------------------------------------------------------------------------------

%%%%%%%%%%%%%%%%%%%%%%%%%%%%%%%%%%%%%%%%%%%%%%%%%%%%%%%%%%%%%%%%%%%%%%%%%%%%%%%%
% You can have multiple style options the legal options ones are:
%
%   centered:	the name and address are centered at the top of the page 
%				(default)
%
%   line:		the name is the left with a horizontal line then the address to
%				the right
%
%   overlapped:	the section titles overlap the body text (default)
%
%   margin:		the section titles are to the left of the body text
%		
%   11pt:		use 11 point fonts instead of 10 point fonts
%
%   12pt:		use 12 point fonts instead of 10 point fonts
%
%%%%%%%%%%%%%%%%%%%%%%%%%%%%%%%%%%%%%%%%%%%%%%%%%%%%%%%%%%%%%%%%%%%%%%%%%%%%%%%%
\let\latexnofiles\nofiles
\let\nofiles\relax
\documentclass[margin]{res}  

% Default font is the helvetica postscript font
\usepackage{helvet}
\usepackage{etaremune}
\usepackage{enumitem,kantlipsum}

% Increase text height
\textheight=700pt

\begin{document}

%-------------------------------------------------------------------------------
%	NAME AND ADDRESS SECTION
%-------------------------------------------------------------------------------
\name{Malena Rice}

% Note that addresses can be used for other contact information:
% -phone numbers
% -email addresses
% -linked-in profile

% Uncomment to add a third address
%\address{Address 3 line 1\\Address 3 line 2\\Address 3 line 3}
%-------------------------------------------------------------------------------

\begin{resume}

%-------------------------------------------------------------------------------
%	EDUCATION SECTION
%-------------------------------------------------------------------------------
\section{\normalfont CONTACT INFORMATION}
{Yale Astronomy Department \hfill Email: malena.rice@yale.edu\\
52 Hillhouse Ave \hfill Web: www.astro.yale.edu/malenarice\\
New Haven, CT 06511\\}

\vspace{-5mm}
\section{\normalfont EDUCATION}
\textbf{Yale University}, New Haven, CT (USA)\\
Ph.D. Astronomy, expected May 2022\\
M.S., M.Phil. Astronomy, May 2020\\
Advisor: Greg Laughlin

\textbf{University of California, Berkeley}, Berkeley, CA (USA) \\
B.A.  Astrophysics with Honors, B.A. Physics, May 2017\\
 With Distinction \\
 Advisor: Gaspard Duch\^{e}ne \\ Honors Thesis: \textit{Debris Disk Analysis with the Gemini Planet Imager}

\section{\normalfont RESEARCH POSITIONS}
\textbf{Graduate Researcher,} Yale University \hfill Aug 2017 - \\
\hspace{5mm} Advisor: Greg Laughlin

\textbf{Undergraduate Researcher,} UC Berkeley \hfill Sep 2015 - May 2017 \\
\hspace{5mm} Advisor: Gaspard Duch\^{e}ne 

\textbf{Undergraduate Researcher,} University College London \hfill May 2016 - Aug 2016
\\
\hspace{5mm} Advisor: Giovanna Tinetti 

\textbf{Undergraduate Researcher,} NASA GSFC \hfill May 2015 - Aug 2015 \\
\hspace{5mm} Advisor: Conor Nixon


%-------------------------------------------------------------------------------
%	FELLOWSHIPS & AWARDS SECTION
%-------------------------------------------------------------------------------
\section{\normalfont FELLOWSHIPS \& AWARDS}
\textbullet{} NSF Graduate Research Fellowship (\$34k/year) \hfill 2017-2022\\
\textbullet{} DDA/AAS Raynor L. Duncombe Student Research Prize \hfill 2020\\
\textbullet{} DPS/AAS Education and Outreach Grant (awarded for astro[sound]bites) \hfill 2020\\
\textbullet{} Pierazzo International Student Travel Award (\$2k) \hfill 2020\\
\textbullet{} NASA CT Space Grant Graduate Research Fellowship (\$8k) \hfill 2019\\
\textbullet{} Binary Asteroids 5 Workshop Travel Award \hfill 2019\\
\textbullet{} Yale Certificate of College Teaching Preparation (CCTP) \hfill 2018\\
\textbullet{} UC Berkeley Regents' and Chancellor's Scholarship (\$10k) \hfill 2013-2017\\
\textbullet{} UC Berkeley Leadership Award (\$2k) \hfill 2016-2017, 2013-2014\\
\textbullet{} UC Berkeley Regents' and Chancellor's Research Fellowship \hfill 2016, 2x\\
\textbullet{} Society of Physics Students (SPS) Travel Award \hfill 2016, 2x\\
\textbullet{} UCL International Students Dean's Summer Student Scholarship (\pounds5k) \hfill 2016\\
\textbullet{} NASA CA Space Grant Undergraduate Research Fellowship \hfill 2016\\
\textbullet{} UC Berkeley Academic Opportunity Fund Award \hfill 2016, 2015\\
\textbullet{} Berkeley Physics Undergraduate Research Scholarship \hfill 2016, 2015\\
%\textbullet{} Royal High School Instrumental Music Merit Scholarship \hfill 2013\\
%\textbullet{} Tom Chiate Scholarship \hfill 2013\\


\section{\normalfont TELESCOPE PROPOSALS}
Keck Observatory (HIRES) - 3 nights awarded (PI) \hfill Yale 2021A\newline
\textit{On the Origins of Exoplanet Spin-Orbit Misalignments}

Keck Observatory (HIRES) - 2 nights awarded (PI) \hfill Yale 2020B\newline
\textit{Non-Transiting Hot Jupiters: Hidden Companions to Known Exoplanets}\\
{Top-ranked proposal by all members of the TAC}

Keck Observatory (HIRES) - 2 nights awarded (CoI) \hfill Yale 2020A\newline
\textit{Non-Transiting Hot Jupiters: Hidden Companions to Known Exoplanets}

Keck Observatory (HIRES) - 2 nights awarded (CoI) \hfill Yale 2020A \newline
\textit{Are Hot Jupiters Dynamically Hot?}

Keck Observatory (HIRES) - 4 nights awarded (CoI) \hfill Yale 2019B \newline
\textit{Non-Transiting Hot Jupiters: Hidden Companions to Known Exoplanets}\\
{Top-ranked proposal by all members of the TAC}

%-------------------------------------------------------------------------------
% Modify the format of each position
\begin{format}
\title{l}\\
\dates{l}\location{r}\\
\body\\
\end{format}
%-------------------------------------------------------------------------------

\section{\normalfont REFEREED PUBLICATIONS}

First-author:
\begin{etaremune}
%\item \textbf{Rice, M.} \& Laughlin, G. 2020 \textit{(in review, PSJ)} %\textit{Exploring Trans-Neptunian Space with TESS: A Targeted Search for Planet Nine in the Galactic Plane}
\item \textbf{Rice, M.} \& Laughlin, G. 2020 \textit{(in press, PSJ)}. \textit{Exploring Trans-Neptunian Space with TESS: A Targeted Search for Planet Nine and Distant TNOs in the Galactic Plane}
\item \textbf{Rice, M.} \& Brewer, J. 2020 \textit{ApJ} 898, 119. \textit{Stellar Characterization of Keck HIRES Spectra with The Cannon}
\item \textbf{Rice, M.} \& Laughlin, G. 2019 \textit{ApJL} 844, L22. \textit{Hidden Planets: Implications from 'Oumuamua and DSHARP}
\item \textbf{Rice, M.} \& Laughlin, G. 2019 \textit{AJ} 158, 19. \textit{The Case for a Large-Scale Occultation Network}
\end{etaremune}

Second-author:
\begin{etaremune}
\item  Duch\^{e}ne, G., \textbf{Rice, M.}, et al. 2020 \textit{AJ}, 159, 251.  \textit{The Gemini Planet Imager View of the HD 32297 Debris Disk}
\item Edwards, B., \textbf{Rice, M.}, Zingales, T., Tessenyi, M., Waldmann, I., Tinetti, G. et al. 2018, \textit{Experimental Astronomy} 47, 29. \textit{Exoplanet Spectroscopy and Photometry with the Twinkle Space Telescope}
\end{etaremune}

Other co-author:
\begin{etaremune}
%\item{Wang, S., Addison, B.C., Winn, J.N., et al. (incl \textbf{Rice, M.}) 2020 (in prep). \textit{K2-232b: A Misaligned Warm Jupiter}}
\item Kosiarek, M., Crossfield, I., Berardo, D., et al. (incl \textbf{Rice, M.}) 2020 \textit{(in review, AJ)}. \textit{Physical Parameters of the Multi-Planet Systems HD 106315 and GJ 9827}
\item  Worku, K., Wang, S., Burt, J., \textbf{Rice, M.}, et al. 2020 \textit{(in review, \textit{AJ})}.  \textit{Revisiting the Full Sets of Orbital Parameters for the XO-3 System: No Evidence for Temporal Variation of the Spin-Orbit Angle}
\item Esposito, T., Kalas, P., Fitzgerald, M.P., et al. (incl \textbf{Rice, M.}) 2020 \textit{AJ} 160, 24. \textit{Debris Disk Results from the Gemini Planet Imager Exoplanet Survey's Polarimetric Imaging Campaign}
\item Blunt, S., Wang, J.,  Angelo, I.,  Ngo, H., et al. (incl \textbf{Rice, M.}) 2020 \textit{AJ} 159, 89. \textit{orbitize!: A Comprehensive Orbit-Fitting Software Package for the High-Contrast Imaging Community}
\item Nixon, C.A., Ansty, T.M., Lombardo, N.A., Bjoraker, G.L., Achterberg, R.K., Annex, A., \textbf{Rice, M.}, et al. 2019 \textit{ApJS} 244, 14. \textit{Cassini Composite Infrared Spectrometer (CIRS) Observations of Titan 2004-2017}
\item Ren, B., Choquet, \'{E.}, Perrin, M.D., Duch\^{e}ne, G., Debes, J.H., Pueyo, L., \textbf{Rice, M.} et. al. 2019 \textit{ApJ} 882, 64. \textit{An Exo-Kuiper Belt with an Extended Halo around HD 191089 in Scattered Light}
\item Esposito, T.M., Duch\^{e}ne, G., Kalas, P., \textbf{Rice, M.}, Choquet, \'{E}., Ren, B., Perrin, M.D. et al. 2018 \textit{AJ} 156, 2. \textit{Direct Imaging of the HD 35841 Debris Disk: A Polarized Dust Ring from Gemini Planet Imager and an Outer Halo from HST/STIS}
\end{etaremune}



\section{\normalfont INVITED SEMINARS \& COLLOQUIA}
CCA Stars \& Exoplanets Meeting \textit{(webinar)} \hfill July 2020 \\
\textit{Stellar Characterization of Keck HIRES Spectra with The Cannon}

Columbia University \textit{(webinar)} \hfill April 2020 \\
\textit{Exploring the Trans-Neptunian Solar System with TESS}

San Francisco State University \textit{(San Francisco, CA -- USA)} \hfill Feb 2020 \\
\textit{Mapping the Trans-Neptunian Solar System}

UC Berkeley CIPS Seminar \textit{(Berkeley, CA -- USA)} \hfill Feb 2020 \\
\textit{Mapping the Trans-Neptunian Solar System}

Keck Observatory \textit{(Waimea, HI -- USA)} \hfill Nov 2019 \\
\textit{Probing the Unknown Fringes of the Solar System}

NASA Astrobiology Institute Extended Science Talk \textit{(webinar)} \hfill Oct 2019 \\
\textit{Interstellar Objects and DSHARP Point to $10^{11}$ Hidden Planets}

University of Hong Kong \textit{(Pok Fu Lam -- Hong Kong)} \hfill Aug 2019 \\
\textit{Probing the Unknown Fringes of the Solar System}

University of Chicago \textit{(Chicago, IL -- USA)} \hfill April 2019 \\
\textit{Pushing Boundaries: Large-Scale Projects in Planetary Science}

%Yale Exoplanet Seminar \textit{(New Haven, CT -- USA)} \hfill March 2019 \\
%\textit{Pushing Boundaries: Large-Scale Projects in Planetary Science}

%UC Berkeley Astronomy Lunch Talk \textit{(Berkeley, CA -- USA)} \hfill April 2017 \\
%\textit{The Gemini Planet Imager View of the HD 32297 Debris Disk System}


\section{\normalfont CONFERENCE RESEARCH \\TALKS}
DPS 52 Conference \textit{(online)} \hfill Oct 2020 \\
Europlanet Science Congress (EPSC) 2020 \textit{(online)} \hfill Sep 2020 \\
AAS Division of Dynamical Astronomy Meeting \#51 \textit{(online)} \hfill Aug 2020 \\
Binary Asteroids V \textit{(Fort Collins, CO -- USA)} \hfill Sep 2019 \\
%\textit{Constraints on the Scattered Disk Population from Binary Cold Classical KBOs} \\
Extreme Solar Systems IV \textit{(Reykjav\'{i}k, Iceland)} \hfill Aug 2019 \\
%\textit{'Oumuamua and DSHARP Point to $10^{11}$  Hidden Planets} \\
Great Barriers in Planet Formation Disc-ussion \textit{(Melbourne, Australia)} \hfill July 2019 \\
%\textit{The Case for a Large-Scale Occultation Network} \\
Emerging Researchers in Exoplanet Science V \textit{(Ithaca, NY -- USA)} \hfill June 2019 \\
%\textit{The Case for a Large-Scale Occultation Network} \\
%Yale University \textit{(New Haven, CT -- USA)} \hfill April 2019 \\
%\textit{The Case for a Large-Scale Occultation Network} \\
Large Surveys with Small Telescopes \textit{(Bamberg, Germany)} \hfill March 2019 \\ 
%\textit{The Case for a Large-Scale Occultation Network} \\
Boston Area Exoplanets \#5 \textit{(Boston, MA -- USA)} \hfill Jan 2019 \\
%\textit{An Occultation Network as a Detector of Distant Solar System Objects}
AAS General Meeting \#233 \textit{(Seattle, WA -- USA)} \hfill Jan 2019
%\textit{An Occultation Network as a Detector of Distant Solar System Objects} \\
%Yale University \textit{(New Haven, CT -- USA)} \hfill April 2018 \\
%\textit{Slowly-Growing Spiral Mode Instabilities in Protostellar Disks} \\
%Leiden Observatory \textit{(Leiden, Netherlands)} \hfill Feb 2017 \\
%\textit{The Gemini Planet Imager View of the HD 32297 Debris Disk System} \\
%University College London \textit{(London, UK)} \hfill Aug 2016 \\
%\textit{Target Analysis for the Twinkle Space Mission}

\section{\normalfont RESEARCH POSTERS}
Exoplanets III \textit{(online)} \hfill July 2020 \\
Asia Oceania Geosciences Society Meeting 2019 \textit{(Singapore)} \hfill July 2019 \\
%\textit{The Case for a Large-Scale Occultation Network}
Great Barriers in Planet Formation \textit{(Palm Cove, Queensland - Australia)} \hfill July 2019 \\ 
%\textit{The Case for a Large-Scale Occultation Network}
2018 International HPC Summer School \textit{(Ostrava, Czech Republic)} \hfill July 2018 \\
%\textit{Slowly-Growing Spiral Mode Instabilities in Protostellar Disks.}
Exoplanets II \textit{(Cambridge, UK)} \hfill July 2018 \\
%\textit{Slowly-Growing Spiral Mode Instabilities in Protostellar Disks.}
Emerging Researchers in Exoplanet Science IV \textit{(State College, PA - USA)} \hfill July 2018 \\
%\textit{Slowly-Growing Spiral Mode Instabilities in Protostellar Disks.}
AAS 231st Meeting \textit{(National Harbor, MD - USA)} \hfill Jan 2018 \\
%\textit{Slowly-Growing Spiral Mode Instabilities in Protostellar Disks.}
2017 BPURS Poster Presentation \textit{(Berkeley, CA - USA)} \hfill March 2017 \\
%\textit{The Gemini Planet Imager View of the HD 32297 Debris Disk System.} \\ 
%\begin{itemize}
%    \item Event for scholarship recipients to present their work.
%\end{itemize}
%\vspace{-3mm}
AAS 229th Meeting \textit{(Grapevine, TX - USA)} \hfill Jan 2017 \\
%\textit{The Gemini Planet Imager View of the HD 32297 Debris Disk System.}
Conference for Undergraduate Women in Physics \textit{(Los Angeles, CA - USA)} \hfill Jan 2017 \\
%\textit{The Gemini Planet Imager View of the HD 32297 Debris Disk System.}
DPS 48 / EPSC 11 Conference \textit{(Pasadena, CA - USA)} \hfill Oct 2016 \\
%\textit{Target Analysis for the Twinkle Space Mission.}
Exoplanets I \textit{(Davos, Switzerland)} \hfill July 2016 \\
%\textit{Modeling Debris Disk HD 32297 Using GPI Direct Imaging}
UC Berkeley Undergrad. Astr. Research Showcase \textit{(Berkeley, CA - USA)} \hfill April 2016 \\
%\textit{Modeling Debris Disk HD 32297 Using Data from the Gemini Planet Imager (GPI).}
2016 BPURS Poster Presentation \textit{(Berkeley, CA - USA)} \hfill March 2016 \\
%\textit{Modeling Debris Disk HD 32297 Using Data from the Gemini Planet Imager (GPI).} \\ 
%\begin{itemize}
%    \item Event for scholarship recipients to present their work. 
%\end{itemize}
%\vspace{-3mm}
NASA GSFC Poster Session \textit{(Greenbelt, Maryland - USA)} \hfill July 2015
%\textit{Water Abundance in the Stratospheres of Saturn and Titan Based on Cassini CIRS Infrared Spectra.}


\section{\normalfont OUTREACH TALKS}
Lakeside School Women in STEM Lecture Series (20 min; invited) \hfill Nov 2020 \\
Las Cruces Public Schools Scientist Highlight (1 hr; invited) \hfill Oct 2020 \\
Yale Exploring Science (20 min; invited) \hfill June 2020 \\
MathCounts Girls' Science Day (30 min; invited keynote) \hfill Dec 2018 \\
Yale Open Labs (20 min) \hfill Nov 2018 \\
Astronomy on Tap New Haven (20 min) \hfill Sep 2018

\section{\normalfont TEACHING APPOINTMENTS}
McDougal/Poorvu Graduate Writing Fellow \hfill Jan 2020-
\begin{itemize}
    \item I serve as a scientific writing consultant for graduate students and postdocs at the Yale Graduate Writing Lab. I run oral and written communication workshops, lead NSF GRFP peer review groups, and conduct one-on-one consulting sessions for abstracts, grant/fellowship proposals, and other academic writing.
\end{itemize}
\vspace{-3mm}
McDougal Graduate Teaching Fellow \hfill Aug 2018-
\begin{itemize}
    \item I am a Teaching Fellow at the Yale Poorvu Center for Teaching and Learning, where I run 10-12 pedagogy workshops per year for Yale graduate students and postdocs and assist with Teaching at Yale Day and Yale's Spring Teaching Forum.
\end{itemize}
\vspace{-3mm}
CIRTL Associate \hfill Aug 2019-
\begin{itemize}
    \item The Center for the Integration of Research, Teaching, and Learning (CIRTL) is an NSF Center for Learning and Teaching developed to advance effective teaching practices for diverse student audiences in STEM higher education.
\end{itemize}
\vspace{-3mm}
Yale Poorvu Center Student Advisory Committee Member \hfill 2019-2020 \\
Teaching Fellow - Astronomy 105, Yale University \hfill Spring 2018, Fall 2018
\begin{itemize}
    \item Introductory order-of-magnitude class, led by Prof. Greg Laughlin.
\end{itemize}
\vspace{-3mm}
Teaching Fellow - Astronomy 130, Yale University \hfill Fall 2017
\begin{itemize}
    \item Introductory exoplanets/astrobiology class, led by Prof. Debra Fischer.
\end{itemize}
\vspace{-3mm}
Student Instructor - Astronomy 120, UC Berkeley \hfill Fall 2016
\begin{itemize}
    \item Upper-division (3rd- and 4th- year undergraduate) optical and infrared astronomy laboratory for Astrophysics majors, led by Dr. Gaspard Duch\^{e}ne.
\end{itemize}
\vspace{-3mm}
ScoreBeyond Tutor \hfill 2016-2018
\begin{itemize} 
    \item SAT/ACT tutoring with private company \textit{ScoreBeyond}; developed lesson plans and guided students through problems and test-taking skills. 600+ tutoring hours completed.
\end{itemize}
\vspace{-3mm}
Academic Tutor - Independent \hfill 2011-2018
\begin{itemize}
    \item Volunteer and paid positions tutoring students in a variety of topics.
\end{itemize} 

\section{\normalfont SELECTED OUTREACH}
Astro[sound]bites Podcast - Co-Founder/Co-Host \hfill 2019-
\begin{itemize}
    \item Co-founder and co-host of the astro[sound]bites podcast, an audio spinoff of the Astrobites blog. Three graduate students discuss recently published astronomy research results.
\end{itemize}
\vspace{-3mm} 

Astronomy on Tap New Haven - Head Coordinator \hfill 2018-
\begin{itemize}
    \item Primary organizer of the New Haven branch of Astronomy on Tap, an outreach program to engage our local community by conveying current astronomy research.
\end{itemize}
\vspace{-3mm}

Leitner Family Observatory and Planetarium - Presenter \hfill 2017- 
\begin{itemize}
    \item I am a regular presenter for weekly public planetarium shows at Yale's campus planetarium, the LFOP.
\end{itemize}
\vspace{-3mm}

Yale Girls' Science Investigations - Regular Volunteer \hfill 2017- \\
Open Labs at Yale - Regular Volunteer \hfill 2017- 

\section{\normalfont PROFESSIONAL SERVICE}
ATHENA by WiSTEM - Research Mentor \hfill 2020 %https://athenabywistem.wixsite.com/athena
\begin{itemize}
    \item I served as a research mentor in the ATHENA program to support minority and underprivileged female high school students in STEM.
\end{itemize}
\vspace{-3mm} 

Yale Undergraduate Research Journal (YURJ) -- Reviewer \hfill 2020 \\

\vspace{-7mm}
National Fund for Sci. and Tech. Dev., Chile -- External Reviewer \hfill 2019 %FONDECYT


\section{\normalfont DEPARTMENT \& UNIVERSITY LEADERSHIP}

Yale Exoplanets \& Stars Seminar Coordinator \hfill 2020- \\
Yale Astro Sibs Program - Co-Founder/Coordinator \hfill 2018-
\begin{itemize}
    \item Developed and lead a mentorship program between graduate students/postdocs and undergraduates in the Yale Astronomy Department.
\end{itemize}
\vspace{-3mm}
Yale ACDC - Co-Founder/Board Member \hfill 2018-
\begin{itemize}
    \item Founded the Yale Astronomy Climate and Diversity Committee (ACDC) to support inclusivity and address climate-related concerns in the department. Co-lead of the Sub-Committee for Undergraduate-Based Affairs (SCUBA)
\end{itemize}
\vspace{-3mm}
UC Berkeley Undergrad. Astronomy Society - Founder/Head Coordinator \hfill 2015-2017 
\begin{itemize}
    \item Founded and developed the undergraduate society for astrophysics majors at UC Berkeley; provided professional development events and networking opportunities for all undergraduate astronomy majors. Programs included an annual UC Berkeley undergraduate astronomy research showcase, bi-weekly undergraduate socials, monthly departmental socials, graduate school/internship application workshops, and visiting scientist events.
\end{itemize}
\vspace{-3mm}
UC Berkeley Study Abroad Student Ambassador \hfill 2015-2017 
\begin{itemize}
    \item Advocated study abroad programs with a focus on STEM majors and international collaboration.
\end{itemize}
\vspace{-3mm}
UC Berkeley Astronomy Mentoring Program - Undergraduate Coordinator \hfill 2016-2017 
\begin{itemize}
    \item Developed and led a mentorship program between graduate students/postdocs and undergraduates in the UC Berkeley Astronomy Department.
\end{itemize}
\vspace{-3mm}
UC Berkeley AstroCDS - Undergraduate Coordinator \hfill 2016-2017 
\begin{itemize}
    \item Revived and led the UC Berkeley Astronomy Career Development Seminar (AstroCDS) program, which organizes informal talks and dinners with Berkeley Astronomy PhDs in industry.
\end{itemize}
\vspace{-3mm}
Space Exploration Society at Berkeley (SESB) - Treasurer \hfill 2014-2015


\section{\normalfont INVITED PANELS}

Yale Astronomy Summer Undergraduate Program: Graduate School Panel \hfill July 2020\\
\vspace{-8mm}

Wellesley College: Graduate School Panel \hfill May 2020\\
\vspace{-8mm}

Yale Graduate Writing Lab: Writing a Prospectus in the Sciences \hfill Feb 2020\\
\vspace{-8mm}

Yale SACNAS/STARS II: Applying to Graduate School \hfill Oct 2019


\section{\normalfont OBSERVING EXPERIENCE}

\textbullet{} Keck I (10 m), HIRES - W.M. Keck Observatory, Hawaii: 13.5 nights \\
\textbullet{} Nickel Telescope (1 m) - Lick Observatory, California: 1 night \\
\textbullet{} Leuschner Telescope (30 inch) - Lafayette Observatory, California: 1 night


\section{\normalfont WORKSHOPS \& SUMMER SCHOOLS}
Astro Hack Week \textit{(online)} \hfill Sep 2020 \\
WFIRST Science Jamboree \textit{(New York City, NY - USA)} \hfill March 2020 \\
TESS Ninja 3: Expanding the Science of TESS \textit{(Sydney, Australia)} \hfill Feb 2020 \\
AAS Hack Together Day \textit{(Seattle, WA - USA)} \hfill Jan 2019 \\
AAS Ambassadors Workshop \textit{(Seattle, WA - USA)} \hfill Jan 2019 \\
La Serena School for Data Science \textit{(La Serena, Chile)} \hfill Aug 2018 \\
International HPC Summer School \textit{(Ostrava, Czech Republic)} \hfill July 2018 \\
International Summer School in Astrobiology \textit{(Santander, Spain)} \hfill June 2017 \\
Brave New Worlds - Lake Como School of Advanced Studies \textit{(Como, Italy)} \hfill May 2016

\section{\normalfont PROFESSIONAL SOCIETIES}
Yale Women in Physics (WiP) \hfill 2019- \\
American Astronomical Society (AAS; divisions DPS and DDA) \hfill 2016- \\
American Physical Society (APS) \hfill 2016-2017 \\
UC Berkeley Society of Women in the Physical Sciences (SWPS) \hfill 2014-2017 \\
UC Berkeley Society of Physics Students (SPS) \hfill 2014-2017 \\
UC Berkeley Regents' and Chancellor's Scholars Association (RCSA) \hfill2013-2017 \\
\vspace{-10mm}

%-------------------------------------------------------------------------------


\end{resume}
\(\)\end{document}