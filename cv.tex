%%%%%%%%%%%%%%%%%%%%%%%%%%%%%%%%%%%%%%%%%%%%%%%%%%%%%%%%%%%%%%%%%%%%%%%%%%%%%%%%
% Medium Length Graduate Curriculum Vitae
% LaTeX Template
% Version 1.2 (3/28/15)
%
% This template has been downloaded from:
% http://www.LaTeXTemplates.com
%
% Original author:
% Rensselaer Polytechnic Institute 
% (http://www.rpi.edu/dept/arc/training/latex/resumes/)
%
% Modified by:
% Daniel L Marks <xleafr@gmail.com> 3/28/2015
%
% Important note:
% This template requires the res.cls file to be in the same directory as the
% .tex file. The res.cls file provides the resume style used for structuring the
% document.
%
%%%%%%%%%%%%%%%%%%%%%%%%%%%%%%%%%%%%%%%%%%%%%%%%%%%%%%%%%%%%%%%%%%%%%%%%%%%%%%%%

%-------------------------------------------------------------------------------
%	PACKAGES AND OTHER DOCUMENT CONFIGURATIONS
%-------------------------------------------------------------------------------

%%%%%%%%%%%%%%%%%%%%%%%%%%%%%%%%%%%%%%%%%%%%%%%%%%%%%%%%%%%%%%%%%%%%%%%%%%%%%%%%
% You can have multiple style options the legal options ones are:
%
%   centered:	the name and address are centered at the top of the page 
%				(default)
%
%   line:		the name is the left with a horizontal line then the address to
%				the right
%
%   overlapped:	the section titles overlap the body text (default)
%
%   margin:		the section titles are to the left of the body text
%		
%   11pt:		use 11 point fonts instead of 10 point fonts
%
%   12pt:		use 12 point fonts instead of 10 point fonts
%
%%%%%%%%%%%%%%%%%%%%%%%%%%%%%%%%%%%%%%%%%%%%%%%%%%%%%%%%%%%%%%%%%%%%%%%%%%%%%%%%
\let\latexnofiles\nofiles
\let\nofiles\relax
\documentclass[margin]{res}  

% Default font is the helvetica postscript font
\usepackage{helvet}
\usepackage{etaremune}
\usepackage{enumitem, kantlipsum}
\usepackage[dvipsnames]{xcolor}
\usepackage[hidelinks, colorlinks=true, allcolors=NavyBlue]{hyperref}

\newcommand{\comment}[1]{}
\newcommand{\paper}[2]{\href{#1}{\sc #2}}
%\newcommand{\link}[1]{}
%\newcommand{\paper}[2]{\href{#1}{\sc \color{RoyalBlue} #2}}

% Increase text height
\textheight=700pt

\begin{document}

\name{John Franklin Crenshaw}
%-------------------------------------------------------------------------------

\begin{resume}

%-------------------------------------------------------------------------------
%	CONTACT INFO
%-------------------------------------------------------------------------------
\section{Contact Information}
{Email: jfc20@uw.edu            \hfill University of Washington Physics Dept \\
Web: \href{https://jfcrenshaw.github.io}{jfcrenshaw.github.io}       \hfill Box 351560 \\
ORCID: \href{https://orcid.org/0000-0002-2495-3514} {0000-0002-2495-3514}     \hfill Seattle, WA 98195}

%-------------------------------------------------------------------------------
%	EDUCATION SECTION
%-------------------------------------------------------------------------------
\section{Education}

\textbf{University of Washington}, Seattle, WA USA\\
Ph.D. in Physics, expected May 2025\\
M.S., Physics, December 2020\\
Advisor: Andrew Connolly

\textbf{Duke University}, Durham, NC USA \\
B.S. in Physics, May 2019\\
\textit{summa cum laude} with Highest Distinction \\
Advisor: Kate Scholberg \\
Thesis: \href{https://jfcrenshaw.github.io/assets/thesis-duke.pdf}{Sensitivity of the Helium and Lead Observatory to Core-Collapse \\ \hspace*{11.4mm} Supernova Neutrino Bursts}

%-------------------------------------------------------------------------------
%	RESEARCH EXPERIENCE
%-------------------------------------------------------------------------------
\section{Research Experience}

\textbf{Graduate Research Assistant} \hfill Aug 2019 -- \\
DiRAC Institute, University of Washington \\
Vera C. Rubin Observatory \\
Dark Energy Science Collaboration (DESC) \\
Informatics and Statistics Science Collaboration (ISSC) \\
Advisor: Andrew Connolly

\textbf{Undergraduate Research Assistant} \hfill Aug 2016 -- May 2019 \\
Duke University, Neutrino and Cosmology Group \\
HALO Supernova Neutrino Detector \\
Advisor: Kate Scholberg 

\textbf{Undergraduate Research Assistant}  \hfill May - Aug 2018 \\
Karlsruhe Institute of Technology, Institute for Nuclear Physics \\
IceTop Cosmic Ray Detector \\
Advisor: Andreas Haungs

%-------------------------------------------------------------------------------
%	FELLOWSHIPS & AWARDS
%-------------------------------------------------------------------------------
\section{Fellowships \& Awards}
Rubin Observatory ISSC Ambassador \hfill 2021 -- 2022 \\
DOE Scholar \hfill 2021 \\
NSF Graduate Research Fellowship Honorable Mention \hfill 2021 \\
Duke Faculty Scholar \hfill 2018 -- 2019 \\
Daphne Chang Memorial Award, Duke Physics Department \hfill 2019 \\
Highest Distinction for Undergraduate Thesis Research \hfill 2019 \\
DAAD RISE Research Exchange Scholarship \hfill 2018 \\

%-------------------------------------------------------------------------------
%	FIRST AUTHOR PUBLICATIONS
%-------------------------------------------------------------------------------
\section{First Author Publications}

\begin{etaremune}[leftmargin=0pt]

\item \paper{https://ui.adsabs.harvard.edu/abs/2020AJ....160..191C/abstract}{Learning Spectral Templates for Photometric Redshift Estimation from Broadband Photometry} \\
\textbf{Crenshaw, J.F.} \& Connolly, A.J. 2020 \textit{AJ}, 160, 191.

\end{etaremune}

%-------------------------------------------------------------------------------
%	OTHER PUBLICATIONS
%-------------------------------------------------------------------------------
\section{Co-Author Publications}

\begin{etaremune}[leftmargin=0pt]

\item \paper{https://ui.adsabs.harvard.edu/abs/2022arXiv220212775S/abstract}{The Sensitivity of GPz Estimates of Photo-z Posterior PDFs to Realistically Complex Training Set Imperfections} \\
Stylianou, N., Malz, A., Hatfield, P., \textbf{Crenshaw, J.F.}, Gschwend, J. \textit{PASP in press} (2022)

\item \paper{https://arxiv.org/abs/2104.08229}{An information-based metric for observing strategy optimization, demonstrated in the context of photometric redshifts with applications to cosmology} \\
Malz, A.I., Lanusse, F., \textbf{Crenshaw, J.F.}, Graham, M.L. \textit{arXiv} (2021)

\end{etaremune}

%-------------------------------------------------------------------------------
%	INVITED TALKS
%-------------------------------------------------------------------------------
\section{Invited Talks}

\hangindent=2mm
AAS Astronomers Turned Data Scientists (ATDS) Meeting (\textit{online}) \hfill March 2022 \\
\href{https://docs.google.com/presentation/d/18cWynQW9p8bEPS6NMueZ4mKaEEmNdxUPhsXArGV5Cb8/edit?usp=sharing}{Simulating Astronomical Data with True Posteriors using Normalizing Flows}

\vspace{-3.5mm} \hangindent=2mm
DESC Winter Meeting (\textit{online}) \hfill Feb 2022 \\ \href{https://docs.google.com/presentation/d/179O1gnQpVfbIwZLfi7TlW8gYhP3K4MFBbi_8lAighNs/edit?usp=sharing}{Deep Generative Modeling for the Photo-z RAIL Pipeline}

\vspace{-3.5mm} \hangindent=2mm
Gruen Weak Lensing Group, KIPAC, SLAC National Lab (\textit{online}) \hfill Sep 2020 \\ \textit{Deconvolving Galaxy Spectra from Broadband Photometry} \\

%-------------------------------------------------------------------------------
%	CONTRIBUTED TALKS
%-------------------------------------------------------------------------------
\section{Contributed \\Talks}

DESC Winter Meeting (\textit{online}) \hfill Feb 2021 \\
Rubin Observatory Project \& Community Workshop (\textit{online}) \hfill July 2020 \\
DESC Winter Meeting (\textit{Tucson, AZ}) \hfill Jan 2020 \\

%-------------------------------------------------------------------------------
%	POSTERS
%-------------------------------------------------------------------------------
\section{Research \\Posters}

AAS 238th Meeting (\textit{online}) \hfill June 2021 \\
SCMA VII Meeting (\textit{online}) \hfill June 2021 \\
Duke Physics Undergraduate Research Symposium (\textit{Durham, NC}) \hfill April 2019 \\
5th Joint Meeting of APS and Physical Society of Japan (\textit{Waikoloa, HI}) \hfill Oct 2018 \\
Neutrino 2018 (\textit{Heidelberg, Germany}) \hfill June 2018 \\

%-------------------------------------------------------------------------------
%	TEACHING
%-------------------------------------------------------------------------------
\section{Teaching Experience}

Undergraduate Reading Course Instructor, $\Lambda$CDM Cosmology \hfill 2021 -- \\
Teaching Assistant, Intro Physics Courses, Duke University \hfill 2016 -- 2019 \\
Physics and Math Tutor, Duke University \hfill 2016 -- 2019 \\

%-------------------------------------------------------------------------------
%	OUTREACH
%-------------------------------------------------------------------------------
\section{Outreach}

STEM Pals organizer \& pedagogical simulation developer  \hfill 2021 \\
Duke University Teaching Observatory, volunteer \hfill 2018 -- 2019 \\
Duke University Public Lecture: \textit{Where Did We Come From} \\
\hspace*{4mm} \textit{and Are We Alone: cosmic origins and the search for life} \hfill January 2018



%-------------------------------------------------------------------------------
%	LEADERSHIP
%-------------------------------------------------------------------------------
\section{Service \& Leadership}

Undergraduate Reading Course Leadership Committee, UW \hfill 2022 -- \\
Physicists for Inclusion and Equity (PIE) Officer, UW \hfill 2020 -- 2021 \\
Departmental Review Student Committee, Duke Physics Department \hfill 2018 \\

%-------------------------------------------------------------------------------
%	OUTREACH
%-------------------------------------------------------------------------------
%\section{\normalfont OBSERVING EXPERIENCE}

%Apache Point Observatory - 3.5 m, DIS: 3 nights \\

%-------------------------------------------------------------------------------
%	PROFESSIONAL SOCIETIES
%-------------------------------------------------------------------------------
\section{Professional Societies}

American Astronomical Society (AAS) \\
American Physical Society (APS) \\
Phi Beta Kappa \\
Duke Society of Physics Students (SPS) \\

%-------------------------------------------------------------------------------

\vfill
\strut \hfill
Last updated: \today

\end{resume}
\(\)\end{document}