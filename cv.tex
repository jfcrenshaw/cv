%%%%%%%%%%%%%%%%%%%%%%%%%%%%%%%%%%%%%%%%%%%%%%%%%%%%%%%%%%%%%%%%%%%%%%%%%%%%%%%%
% Medium Length Graduate Curriculum Vitae
% LaTeX Template
% Version 1.2 (3/28/15)
%
% This template has been downloaded from:
% http://www.LaTeXTemplates.com
%
% Original author:
% Rensselaer Polytechnic Institute 
% (http://www.rpi.edu/dept/arc/training/latex/resumes/)
%
% Modified by:
% Daniel L Marks <xleafr@gmail.com> 3/28/2015
%
% Important note:
% This template requires the res.cls file to be in the same directory as the
% .tex file. The res.cls file provides the resume style used for structuring the
% document.
%
%%%%%%%%%%%%%%%%%%%%%%%%%%%%%%%%%%%%%%%%%%%%%%%%%%%%%%%%%%%%%%%%%%%%%%%%%%%%%%%%

%-------------------------------------------------------------------------------
%	PACKAGES AND OTHER DOCUMENT CONFIGURATIONS
%-------------------------------------------------------------------------------

%%%%%%%%%%%%%%%%%%%%%%%%%%%%%%%%%%%%%%%%%%%%%%%%%%%%%%%%%%%%%%%%%%%%%%%%%%%%%%%%
% You can have multiple style options the legal options ones are:
%
%   centered:	the name and address are centered at the top of the page 
%				(default)
%
%   line:		the name is the left with a horizontal line then the address to
%				the right
%
%   overlapped:	the section titles overlap the body text (default)
%
%   margin:		the section titles are to the left of the body text
%		
%   11pt:		use 11 point fonts instead of 10 point fonts
%
%   12pt:		use 12 point fonts instead of 10 point fonts
%
%%%%%%%%%%%%%%%%%%%%%%%%%%%%%%%%%%%%%%%%%%%%%%%%%%%%%%%%%%%%%%%%%%%%%%%%%%%%%%%%
\let\latexnofiles\nofiles
\let\nofiles\relax
\documentclass[margin]{res}  

% Default font is the helvetica postscript font
\usepackage{helvet}
\usepackage{etaremune}
\usepackage{enumitem,kantlipsum}

\newcommand{\comment}[1]{}

% Increase text height
\textheight=700pt

\begin{document}

\name{John Franklin Crenshaw}
%-------------------------------------------------------------------------------

\begin{resume}

%-------------------------------------------------------------------------------
%	CONTACT INFO
%-------------------------------------------------------------------------------
\section{\normalfont CONTACT INFORMATION}
{Email: jfc20@uw.edu            \hfill University of Washington Physics Dept \\
Web: jfcrenshaw.github.io       \hfill Box 351560 \\
ORCID: 0000-0002-2495-3514      \hfill Seattle, WA 98195}

%-------------------------------------------------------------------------------
%	EDUCATION SECTION
%-------------------------------------------------------------------------------
\section{\normalfont EDUCATION}

\textbf{University of Washington}, Seattle, WA USA\\
Ph.D. Physics, expected May 2025\\
%M.S., Physics, December 2020\\
Advisor: Andrew Connolly

\textbf{Duke University}, Durham, NC USA \\
B.S. in Physics, May 2019\\
\textit{summa cum laude} with Highest Distinction \\
Advisor: Kate Scholberg \\
Thesis: \textit{Sensitivity of the Helium and Lead Observatory to CCSN Neutrino Bursts}

%-------------------------------------------------------------------------------
%	RESEARCH EXPERIENCE
%-------------------------------------------------------------------------------
\section{\normalfont RESEARCH EXPERIENCE}

\textbf{Graduate Researcher,} DIRAC Institute, University of Washington \hfill Aug 2019 - \\
Dark Energy Science Collaboration (DESC), Vera C. Rubin Observatory \\
Advisor: Andrew Connolly

\textbf{Undergraduate Researcher,} Duke University \hfill Aug 2016 - May 2019 \\
Duke Neutrino and Cosmology Group \\
HALO Supernova Neutrino Detector \\
Advisor: Kate Scholberg 

\textbf{Undergraduate Researcher,} Karlsruhe Institute of Technology \hfill May - Aug 2018 \\
Institute for Nuclear Physics \\
IceTop Cosmic Ray Detector \\
Advisor: Andreas Haungs

%-------------------------------------------------------------------------------
%	FELLOWSHIPS & AWARDS
%-------------------------------------------------------------------------------
\section{\normalfont FELLOWSHIPS \& AWARDS}
\textbullet{} Duke Faculty Scholar \hfill 2018 - 2019 \\
\textbullet{} Daphne Chang Memorial Award, Duke Physics Department \hfill 2019 \\
\textbullet{} Highest Distinction for Undergraduate Thesis Research \hfill 2019 \\
\textbullet{} DAAD RISE Research Exchange Scholarship \hfill 2018 \\

%-------------------------------------------------------------------------------
%	PUBLICATIONS
%-------------------------------------------------------------------------------
\section{\normalfont PUBLICATIONS}

\textbf{Crenshaw, J.F.} \& Connolly, A.J. 2020 \textit{AJ}, 160, 191. \textit{Learning Spectral Templates for Photometric Redshift Estimation from Broadband Photometry}

%-------------------------------------------------------------------------------
%	SEMINARS AND COLLOQUIA
%-------------------------------------------------------------------------------
\section{\normalfont INVITED SEMINARS \& COLLOQUIA}

Gruen Weak Lensing Group, KIPAC, SLAC National Lab (\textit{online}) \hfill Sep 2020 \\
\textit{Deconvolving Galaxy Spectra from Broadband Photometry} \\

%-------------------------------------------------------------------------------
%	CONFERENCE TALKS
%-------------------------------------------------------------------------------
\section{\normalfont CONFERENCE \\TALKS}

Rubin Observatory Project \& Community Workshop (\textit{online}) \hfill July 2020 \\
DESC Winter Meeting (\textit{Tucson, AZ}) \hfill Jan 2020 \\

%-------------------------------------------------------------------------------
%	POSTERS
%-------------------------------------------------------------------------------
\section{\normalfont RESEARCH POSTERS}

Duke Physics Undergraduate Research Symposium (\textit{Durham, NC}) \hfill April 2019 \\
5th Joint Meeting of APS and Physical Society of Japan (\textit{Waikoloa, HI}) \hfill Oct 2018 \\
Neutrino 2018 (\textit{Heidelberg, Germany}) \hfill June 2018 \\

%-------------------------------------------------------------------------------
%	TEACHING
%-------------------------------------------------------------------------------
\section{\normalfont TEACHING EXPERIENCE}

Teaching Assistant, Intro Physics Courses, Duke University \hfill Aug 2016 - May 2020 \\
Physics and Math Tutor, Duke University \hfill Jan 2016 - May 2020 \\

%-------------------------------------------------------------------------------
%	OUTREACH
%-------------------------------------------------------------------------------
%\section{\normalfont SELECTED OUTREACH}

\comment{
Astro[sound]bites Podcast - Co-Founder/Co-Host \hfill 2019-
\begin{itemize}
    \item Co-founder and co-host of the astro[sound]bites podcast, an audio spinoff of the Astrobites blog. Three graduate students discuss recently published astronomy research results.
\end{itemize}
\vspace{-3mm} 

Astronomy on Tap New Haven - Head Coordinator \hfill 2018-
\begin{itemize}
    \item Primary organizer of the New Haven branch of Astronomy on Tap, an outreach program to engage our local community by conveying current astronomy research.
\end{itemize}
\vspace{-3mm}

Leitner Family Observatory and Planetarium - Presenter \hfill 2017- 
\begin{itemize}
    \item I am a regular presenter for weekly public planetarium shows at Yale's campus planetarium, the LFOP.
\end{itemize}
\vspace{-3mm}

Yale Girls' Science Investigations - Regular Volunteer \hfill 2017- \\
Open Labs at Yale - Regular Volunteer \hfill 2017- 

\section{\normalfont PROFESSIONAL SERVICE}
ATHENA by WiSTEM - Research Mentor \hfill 2020 %https://athenabywistem.wixsite.com/athena
\begin{itemize}
    \item I served as a research mentor in the ATHENA program to support minority and underprivileged female high school students in STEM.
\end{itemize}
\vspace{-3mm} 

Yale Undergraduate Research Journal (YURJ) -- Reviewer \hfill 2020 \\

\vspace{-7mm}
National Fund for Sci. and Tech. Dev., Chile -- External Reviewer \hfill 2019 %FONDECYT
}

\section{\normalfont DEPARTMENT \& UNIVERSITY LEADERSHIP}

Physicists for Inclusion and Equity (PIE) Officer, UW \hfill 2020 - \\
Departmental Review Student Committee, Duke Physics Department \hfill 2018 \\

%-------------------------------------------------------------------------------
%	OUTREACH
%-------------------------------------------------------------------------------
\section{\normalfont OBSERVING EXPERIENCE}

Apache Point Observatory - 3.5 m, DIS: 3 nights \\

%-------------------------------------------------------------------------------
%	PROFESSIONAL SOCIETIES
%-------------------------------------------------------------------------------
\section{\normalfont PROFESSIONAL SOCIETIES}

American Astronomical Society (AAS) \\
American Physical Society (APS) \\
Phi Beta Kappa \\
Duke Society of Physics Students (SPS) \\

%-------------------------------------------------------------------------------

\vfill
\strut \hfill
Last updated: \today

\end{resume}
\(\)\end{document}