%%%%%%%%%%%%%%%%%%%%%%%%%%%%%%%%%%%%%%%%%%%%%%%%%%%%%%%%%%%%%%%%%%%%%%%%%%%%%%%%
% Medium Length Graduate Curriculum Vitae
% LaTeX Template
% Version 1.2 (3/28/15)
%
% This template has been downloaded from:
% http://www.LaTeXTemplates.com
%
% Original author:
% Rensselaer Polytechnic Institute 
% (http://www.rpi.edu/dept/arc/training/latex/resumes/)
%
% Modified by:
% Daniel L Marks <xleafr@gmail.com> 3/28/2015
%
% Important note:
% This template requires the res.cls file to be in the same directory as the
% .tex file. The res.cls file provides the resume style used for structuring the
% document.
%
%%%%%%%%%%%%%%%%%%%%%%%%%%%%%%%%%%%%%%%%%%%%%%%%%%%%%%%%%%%%%%%%%%%%%%%%%%%%%%%%

%-------------------------------------------------------------------------------
%	PACKAGES AND OTHER DOCUMENT CONFIGURATIONS
%-------------------------------------------------------------------------------

%%%%%%%%%%%%%%%%%%%%%%%%%%%%%%%%%%%%%%%%%%%%%%%%%%%%%%%%%%%%%%%%%%%%%%%%%%%%%%%%
% You can have multiple style options the legal options ones are:
%
%   centered:	the name and address are centered at the top of the page 
%				(default)
%
%   line:		the name is the left with a horizontal line then the address to
%				the right
%
%   overlapped:	the section titles overlap the body text (default)
%
%   margin:		the section titles are to the left of the body text
%		
%   11pt:		use 11 point fonts instead of 10 point fonts
%
%   12pt:		use 12 point fonts instead of 10 point fonts
%
%%%%%%%%%%%%%%%%%%%%%%%%%%%%%%%%%%%%%%%%%%%%%%%%%%%%%%%%%%%%%%%%%%%%%%%%%%%%%%%%
\let\latexnofiles\nofiles
\let\nofiles\relax
\documentclass[margin, 11pt]{res}  

% Default font is the helvetica postscript font
\usepackage{helvet}
\usepackage{etaremune}
\usepackage{enumitem, kantlipsum}
\usepackage[dvipsnames]{xcolor}
\usepackage[hidelinks, colorlinks=true, allcolors=NavyBlue]{hyperref}

\newcommand{\comment}[1]{}
\newcommand{\paper}[2]{\href{#1}{\sc #2}}

% Increase text height
%\textheight=700pt

\begin{document}

\name{\Large John Franklin Crenshaw}
%-------------------------------------------------------------------------------

\begin{resume}

%-------------------------------------------------------------------------------
%	CONTACT INFO
%-------------------------------------------------------------------------------
\section{Contact Information}
{Email: jfc20@uw.edu            \hfill University of Washington Physics Dept \\
Web: \href{https://jfcrenshaw.github.io}{jfcrenshaw.github.io}       \hfill Box 351560 \\
ORCID: \href{https://orcid.org/0000-0002-2495-3514} {0000-0002-2495-3514}     \hfill Seattle, WA 98195}

%-------------------------------------------------------------------------------
%	EDUCATION SECTION
%-------------------------------------------------------------------------------
\section{Education}

\textbf{University of Washington}, Seattle, WA USA\\
Ph.D. in Physics, expected May 2025\\
M.S., Physics, December 2020\\
Advisor: Andrew Connolly

\textbf{Duke University}, Durham, NC USA \\
B.S. in Physics, May 2019\\
\textit{summa cum laude} with Highest Distinction \\
Advisor: Kate Scholberg


%-------------------------------------------------------------------------------
%	RESEARCH EXPERIENCE
%-------------------------------------------------------------------------------
\section{Research Experience}

\textbf{Dark Energy Science Collaboration (DESC)} \hfill 2019 - present \\
Developing the photometric redshift pipeline for DESC cosmology \\
Deconvolving galaxy spectra from photometry and studying galaxy evolution \\
Photometric measurements of the intergalactic and circumgalactic media \\
Advisor: Andrew Connolly

\textbf{The Vera C. Rubin Observatory} \hfill 2021 - present \\
Commissioning the active optics system \\
Developing deep learning methods to improve telescope wavefront estimation \\
Leading photometric redshift commissioning efforts \\
Advisors: Andrew Connolly and Sandrine Thomas

\textbf{Duke University Neutrino and Cosmology Group} \hfill 2016 - 2019 \\
Simulating core-collapse supernova neutrino bursts \\
Characterizing the sensitivity of the Helium and Lead Observatory (HALO) \\
Advisor: Kate Scholberg

\textbf{Karlsruhe Institute of Technology} \hfill 2018 \\
Studying muon content of cosmic rays detected with the IceTop Array \\
Developing machine learning methods for data analysis \\
Advisor: Andreas Haungs


%-------------------------------------------------------------------------------
%	FELLOWSHIPS & AWARDS
%-------------------------------------------------------------------------------
\section{Fellowships \& Awards}
Rubin Observatory ISSC Ambassador \hfill 2021 - 2022 \\
DOE Scholar \hfill 2021 \\
NSF Graduate Research Fellowship Honorable Mention \hfill 2021 \\
Duke Faculty Scholar \hfill 2018 - 2019 \\
Daphne Chang Memorial Award, Duke Physics Department \hfill 2019 \\
Highest Distinction for Undergraduate Thesis Research \hfill 2019 \\
DAAD RISE Research Exchange Scholarship \hfill 2018 \\

%-------------------------------------------------------------------------------
%	FIRST AUTHOR PUBLICATIONS
%-------------------------------------------------------------------------------
\newpage
\section{First Author Publications}

\begin{etaremune}[leftmargin=0pt]

\item \paper{https://ui.adsabs.harvard.edu/abs/2020AJ....160..191C/abstract}{Learning Spectral Templates for Photometric Redshift Estimation from Broadband Photometry} \\
\textbf{Crenshaw, J.F.} \& Connolly, A.J. 2020 \textit{AJ}, 160, 191.

\end{etaremune}

%-------------------------------------------------------------------------------
%	OTHER PUBLICATIONS
%-------------------------------------------------------------------------------
\section{Co-Author Publications}

\begin{etaremune}[leftmargin=0pt]

\item \paper{https://ui.adsabs.harvard.edu/abs/2022arXiv220602815L/abstract}{The Simulated Catalogue of Optical Transients and Correlated Hosts (SCOTCH)} \\
Lokken, M., Gagliano, A., Narayan, G., Hložek, R., Kessler, R., \textbf{Crenshaw, J. F.}, Salo, L., Alves, C. S., Chatterjee, D., Vincenzi, M., Malz, A. \textit{MNRAS, submitted} (2022)

\item \paper{https://ui.adsabs.harvard.edu/abs/2022arXiv220212775S/abstract}{The Sensitivity of GPz Estimates of Photo-z Posterior PDFs to Realistically Complex Training Set Imperfections} \\
Stylianou, N., Malz, A., Hatfield, P., \textbf{Crenshaw, J.F.}, Gschwend, J. \textit{PASP} (2022)

\item \paper{https://arxiv.org/abs/2104.08229}{An information-based metric for observing strategy optimization, demonstrated in the context of photometric redshifts with applications to cosmology} \\
Malz, A.I., Lanusse, F., \textbf{Crenshaw, J.F.}, Graham, M.L. \textit{arXiv} (2021)

\end{etaremune}

%-------------------------------------------------------------------------------
%	INVITED TALKS
%-------------------------------------------------------------------------------
\section{Invited Talks}

\hangindent=2mm
DESC Summer Meeting (\textit{Chicago}) \hfill Aug 2022 \\
\href{https://docs.google.com/presentation/d/1XIaaf1TzgIAM4Hg7VIUtnKwvI75EPUbHzXDIjFfOA2o/edit?usp=sharing}{Seeing the Forest for the Trees: Detecting a Photometric Lyman-$\alpha$ Signal \\with the Vera Rubin Observatory}

\vspace{-3.5mm} \hangindent=2mm
AAS Astronomers Turned Data Scientists (ATDS) Meeting (\textit{online}) \hfill Mar 2022 \\
\href{https://docs.google.com/presentation/d/18cWynQW9p8bEPS6NMueZ4mKaEEmNdxUPhsXArGV5Cb8/edit?usp=sharing}{Simulating Astronomical Data with True Posteriors using Normalizing Flows}

\vspace{-3.5mm} \hangindent=2mm
Plenary Talk, DESC Winter Meeting (\textit{online}) \hfill Feb 2022 \\ \href{https://docs.google.com/presentation/d/179O1gnQpVfbIwZLfi7TlW8gYhP3K4MFBbi_8lAighNs/edit?usp=sharing}{Deep Generative Modeling for the Photo-z RAIL Pipeline}

\vspace{-3.5mm} \hangindent=2mm
Gruen Weak Lensing Group, KIPAC, SLAC National Lab (\textit{online}) \hfill Sep 2020 \\ \textit{Deconvolving Galaxy Spectra from Broadband Photometry} \\

%-------------------------------------------------------------------------------
%	CONTRIBUTED TALKS
%-------------------------------------------------------------------------------
\section{Contributed \\Talks}

DESC Winter Meeting (\textit{online}) \hfill Feb 2021 \\
Rubin Observatory Project \& Community Workshop (\textit{online}) \hfill Jul 2020 \\
DESC Winter Meeting (\textit{Tucson, AZ}) \hfill Jan 2020 \\

%-------------------------------------------------------------------------------
%	POSTERS
%-------------------------------------------------------------------------------
\section{Research \\Posters}

AAS 238th Meeting (\textit{online}) \hfill Jun 2021 \\
SCMA VII Meeting (\textit{online}) \hfill Jun 2021 \\
Duke Physics Research Symposium (\textit{Durham, NC}) \hfill Apr 2019 \\
5th Joint Meeting of APS and Phys. Soc. of Japan (\textit{Waikoloa, HI}) \hfill Oct 2018 \\
Neutrino 2018 (\textit{Heidelberg, Germany}) \hfill Jun 2018 \\

%-------------------------------------------------------------------------------
%	TEACHING
%-------------------------------------------------------------------------------
\section{Teaching Experience}

\textbf{Reading Course Instructor, University of Washington} \hfill 2021 - present \\
Independently designed syllabi and taught advanced reading courses to undergraduates, including \textit{Tensions in $\Lambda$CDM Cosmology} (2021-2022) and \textit{Gravitational Lensing: From Exoplanets to Large Scale Structure} (2022 -- present).

\textbf{Teaching Assistant, Duke University} \hfill 2016 - 2019 \\
Led lab and discussion sections and gave lectures covering introductory mechanics, fluid dynamics, electromagnetism, and optics.

\textbf{Undergraduate Tutor, Duke University} \hfill 2016 - 2019 \\
Worked as a tutor for introductory physics, modern physics, calculus I - III, and linear algebra.

%-------------------------------------------------------------------------------
%	OUTREACH
%-------------------------------------------------------------------------------
\section{Outreach}

Graduate Student Research Panel, UC Berkeley \hfill Jul 2021 \\
STEM Pals organizer \& pedagogical simulation lead  \hfill 2021 \\
Duke University Teaching Observatory, volunteer \hfill 2018 - 2019 \\
``Queer in Research'' Discussion Panel \hfill Oct 2018 \\
Duke University Public Lecture: \textit{Where Did We Come From} \\
\hspace*{4mm} \textit{and Are We Alone: Cosmic Origins and the Search for Life} \hfill Jan 2018



%-------------------------------------------------------------------------------
%	SERVICE & LEADERSHIP
%-------------------------------------------------------------------------------
\section{Service \& Leadership}

Physics Undergraduate Reading Course \\
\hspace*{4mm} Leadership Committee, University of Washington \hfill 2022 -- present \\
Photo-z Commissioning Organizer, 2022 Rubin Observatory \\ 
\hspace*{4mm} Project and Community Workshop (\textit{Tucson, AZ}) \hfill Aug 2022 \\
Snowmass 2021 Summer Study, Organizing Volunteer \hfill Jul 2022 \\
Physicists for Inclusion and Equity (PIE) Officer, UW \hfill 2020 -- 2021 \\
Departmental Review Student Committee, Duke Physics Department \hfill 2018 \\

%-------------------------------------------------------------------------------
%	OBSERVING
%-------------------------------------------------------------------------------
%\section{\normalfont OBSERVING EXPERIENCE}

%Apache Point Observatory - 3.5 m, DIS: 3 nights \\

%-------------------------------------------------------------------------------
%	PROFESSIONAL SOCIETIES
%-------------------------------------------------------------------------------
\section{Professional Societies}

American Astronomical Society (AAS) \\
American Physical Society (APS) \\
Phi Beta Kappa \\
Duke Society of Physics Students (SPS) \\

%-------------------------------------------------------------------------------

\vfill
\strut \hfill
Last updated: \today

\end{resume}
\(\)\end{document}